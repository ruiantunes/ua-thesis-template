% This file allows the author to manually create a "Notation",
% "Nomenclature", or "List of symbols" section, by having full control
% and design freedom. Also, it has no special package dependencies.
%
% This "notation.tex" file was inspired from other works:
%
% [1] Tiago Almeida (2025).
%     Biomedical question answering: from retrieval to answer generation.
%     http://hdl.handle.net/10773/44153
%
% [2] Ian Goodfellow, Yoshua Bengio, and Aaron Courville (2016).
%     Deep Learning.
%     https://www.deeplearningbook.org
%     https://github.com/goodfeli/dlbook_notation/
%
% [3] Yarin Gal (2016).
%     Uncertainty in deep learning.
%     https://idiscover.lib.cam.ac.uk/permalink/f/t9gok8/44CAM_ALMA21582084170003606
%
% [4] Simon Baker (2017).
%     Semantic text classification for cancer text mining.
%     https://doi.org/10.17863/CAM.23105
%
% Different ways (more or less complex, and more or less automatic) of
% achieving a similar result do exist, but are not implemented in this
% template. For example, see:
%   https://tex.stackexchange.com/questions/348640/how-to-effectively-use-list-of-symbols-for-a-thesis
%   https://www.overleaf.com/learn/latex/Nomenclatures

\begingroup
% Set the glue (vertical space) before a "longtable" to be 0pt.
% https://tex.stackexchange.com/questions/424176/longtable-vertical-space
\setlength{\LTpre}{0pt}

% Make the "longtable" environment to be left-aligned by default.
% https://stackoverflow.com/questions/3345077/latex-left-align-a-table-ie-not-centred-from-the-preamble
\setlength{\LTleft}{0pt}
\setlength{\LTright}{\fill}

% Set the "tocdepth" counter to 0 so only the chapter-level is shown
% at the table of contents.
% https://tex.stackexchange.com/questions/4102/setting-setcountertocdepth-for-an-individual-chapter
\addtocontents{toc}{\protect\setcounter{tocdepth}{0}}
% Set the "secnumdepth" counter to 0 so no section level is numbered.
\setcounter{secnumdepth}{0}

\phantomsection
\addcontentsline{toc}{chapter}{\notationname}
\chapter*{\notationname}
\markboth{\notationname}{}
\label{notation}

This part of the template can be edited to manually create a \textbf{Notation}, \textbf{Nomenclature}, or \textbf{List of symbols} section, as the author may prefer.
I leave below a draft of what it may look like.


\section{Physical constants}
\label{notation:s:physical-constants}
% Adapted from:
% https://www.overleaf.com/learn/latex/Nomenclatures

\begin{longtable}{llr}
$G$ & Gravitational constant     & \SI[group-digits=false]{6.67430e-11}{\meter\cubed\per\kilogram\per\second\squared}\\
$c$ & Speed of light in a vacuum & \SI{299792458}{\meter\per\second}\\
$h$ & Planck constant            & \SI[group-digits=false]{6.62607015e-34}{\joule\per\hertz}\\
\end{longtable}


\section{Mathematical symbols}
\label{notation:s:mathematical-symbols}
% Adapted from:
% https://en.wikipedia.org/wiki/Glossary_of_mathematical_symbols


\subsection{Arithmetic operators}
\label{notation:ss:arithmetic-operators}

\begin{longtable}{ll}
$+$      & plus sign\\
$-$      & minus sign\\
$\times$ & multiplication sign\\
\end{longtable}


\subsection{Equality}
\label{notation:ss:equality}

\begin{longtable}{ll}
$=$       & equality\\
$\neq$    & inequality\\
$\approx$ & approximate equality\\
\end{longtable}


\subsection{Comparison}
\label{notation:ss:comparison}

\begin{longtable}{ll}
$<$    & less-than sign\\
$>$    & greater-than sign\\
$\leq$ & less than or equal to\\
$\geq$ & greater than or equal to\\
\end{longtable}


\section{Numbers}
\label{notation:s:numbers}

\begin{longtable}{ll}
$a$         & A scalar\\
$\vec{a}$   & A vector\\
$A$         & A matrix\\
\end{longtable}


\section{Operations}
\label{notation:s:operations}
% https://tex.stackexchange.com/questions/107186/how-to-write-norm-which-adjusts-its-size
% https://tex.stackexchange.com/questions/30619/what-is-the-best-symbol-for-vector-matrix-transpose

\begin{longtable}{ll}
$|\vec{a}|$                      & $L_1$ norm of vector $\vec{a}$\\
$\left\lVert\vec{a}\right\rVert$ & $L_2$ norm of vector $\vec{a}$\\
$A^\top$                         & Transpose of matrix $A$\\
\end{longtable}


\section{Other symbols}
\label{notation:s:other-symbols}

Add here some other symbols as needed.


\section{Miscellaneous}
\label{notation:s:miscellaneous}

Other typography rules, conventions, or definitions.


% The \cleardoublepage command is necessary to make sure that the
% page numbering remains correct.
\cleardoublepage
% Reset the "tocdepth" and "secnumdepth" counters to the original
% values.
\addtocontents{toc}{\protect\setcounter{tocdepth}{\mytocdepth}}
\setcounter{secnumdepth}{\mysecnumdepth}

\endgroup
