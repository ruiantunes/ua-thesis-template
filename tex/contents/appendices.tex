\begin{appendices}
\crefalias{chapter}{appendix}
\chapter{Appendix example}
\label{a1}

This is the first appendix.


\section{A section example}

Similarly to a chapter we can add sections, subsections, and so on...

\chapter{A second example of an appendix}
\label{a2}

This is the second appendix.

\end{appendices}
