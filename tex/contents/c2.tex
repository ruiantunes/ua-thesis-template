\chapter{Tips and examples}
\label{c2}

This chapter presents some basic tips and a few examples on how to use \LaTeX.


\section{\TeX\ engines}
\label{c2:s:tex-engines}

There are several \TeX\ engines.
In short, these are used to compile the (La)TeX source code to generate the output file (for example, a \as{pdf}).
To know more about these, I suggest you to check these articles:

\begin{itemize}
\item
The TeX family tree: LaTeX, pdfTeX, XeTeX, LuaTeX and ConTeXt.\\
\url{https://www.overleaf.com/learn/latex/Articles/The_TeX_family_tree:_LaTeX,_pdfTeX,_XeTeX,_LuaTeX_and_ConTeXt}

\item
Choosing a LaTeX Compiler.\\
\url{https://www.overleaf.com/learn/latex/Choosing_a_LaTeX_Compiler}

\item
Are there benefits to use XeTeX or LuaTeX if one is to write documents mainly in English?\\
\url{https://tex.stackexchange.com/questions/548467/are-there-benefits-to-use-xetex-or-luatex-if-one-is-to-write-documents-mainly-in}

\item
Why choose LuaLaTeX over XeLaTeX.\\
\url{https://tex.stackexchange.com/questions/126206/why-choose-lualatex-over-xelatex}

\item
Differences between LuaTeX, ConTeXt and XeTeX.\\
\url{https://tex.stackexchange.com/questions/36/differences-between-luatex-context-and-xetex}

\end{itemize}

In this template, support for both pdf\TeX\ and Lua\TeX\ engines has been guaranteed, but I encourage you to use the latter because it is more powerful for typefaces: it supports TrueType and OpenType standards.


\subsection{Compiler automatic detection}

Automatic detection of the \LaTeX\ compiler in use:

\ifPDFTeX
{\color{red} pdf\TeX\ is being used.}

Consider changing to Lua\TeX, which is the recommended compiler for this template.
\fi

\ifLuaTeX
{\color{Green4} Lua\TeX\ is being used.}

You are using the recommended compiler.
\fi


\section{Basic tips}
\label{c2:s:basic-tips}

\begin{itemize}
\item
Use the \verb+%+ (percentage) symbol to comment (ignore) lines in the source code.
\item
The \verb+\\+ and \verb+\newline+ commands are similar, but are not the same.\\
\url{https://tex.stackexchange.com/questions/27028/what-is-the-difference-between-newline-and}.
\item
The \verb+\cleardoublepage+ command forces the next content to start in an odd page.
\item
The tilde character (\verb+~+) inserts a non-breaking space.
Use it before citing a reference to avoid breaking the line: \verb+an example~\cite{label}+.
\item
The current font size is \myfontsize.
\item
Use the \verb+longtable+ environment for tables spanning multiple pages.
\item
The grave accent \`{} and the apostrophe \textquotesingle\ are the correct symbols to make quotations: ``this is an example''.
\end{itemize}


\section{How to compile the document faster}
\label{c2:s:how-to-compile-the-document-faster}

Some techniques can be employed to compile the document faster (note that these do not apply for generating the final document).
In the \verb+main.tex+ file:

\begin{itemize}
\item Use the \verb+draft+ option in the \verb+graphicx+ package (images will be replaced by placeholders of the same size);
\item Comment the \verb+first-pages.tex+ and \verb+notation.tex+ files;
\item Compile only the chapter that you're currently working with (the other chapters can be commented);
\item You may also comment the references, appendices, and annexes.
\end{itemize}


\section{The chapter style}
\label{c2:s:the-chapter-style}

Using the \verb+titlesec+ package you can build your own chapter style.
Instead, for ease of use, you may use one of the styles that are already provided in this template:

\begin{itemize}

\item
Default style:\\
\verb+% Adapted from:
% https://tug.ctan.org/macros/latex/contrib/titlesec/titlesec.pdf#subsection.8.2

\titleformat{\chapter}[display]
  {\normalfont\huge\bfseries}{\chaptertitlename\ \thechapter}{20pt}{\Huge}

\titlespacing*{\chapter}{0pt}{50pt}{40pt}
+

\item
Veelo chapter style:\\
\verb+% Veelo chapter style. Adapted from:
% https://tex.stackexchange.com/questions/487917/setting-chapter-style-with-bar
% https://tex.stackexchange.com/questions/177144/chapter-style-report-class
% Note that you can change \rule{30pt} to another width (such as 28pt
% or 32pt if you prefer a thinner or thicker bar).
\newcommand*\chapnamefont{\normalfont\LARGE\MakeUppercase}
\newcommand*\chapnumfont{\normalfont\Huge}
\newcommand*\chaptitlefont{\normalfont\Huge\bfseries}

\newlength\beforechapskip
\newlength\midchapskip
\setlength\midchapskip{\paperwidth}
\addtolength\midchapskip{-\textwidth}
\addtolength\midchapskip{-\oddsidemargin}
\addtolength\midchapskip{-1in}
% Also adjust this value accordingly to your preference.
\setlength\beforechapskip{17mm}

\titleformat{\chapter}[display]
  {\normalfont\filleft}
  {{\chapnamefont\chaptertitlename}%
    \rlap{\hspace{.8em}%
      \makebox[\dimexpr\oddsidemargin+\hoffset+1in][s]{{\resizebox{!}{\beforechapskip}{\chapnumfont\thechapter}}\hfill%
      \rule{30pt}{\beforechapskip}}%
    }%
  }%
  {25pt}
  {\chaptitlefont}
\titlespacing*{\chapter}
  {0pt}{40pt}{40pt}
+

\item
Centered title within two rules:\\
\verb+% Adapted from:
% https://pt.overleaf.com/learn/latex/Sections_and_chapters
\titleformat
  {\chapter} % command
  [display] % shape
  {\bfseries\LARGE} % format
  {\chaptertitlename\ \thechapter} % label
  {0.5ex} % sep
  {
    \rule{\textwidth}{1pt}
    \vspace{1ex}
    \centering
    \huge
  } % before-code
  [
    \vspace{-2.0ex}%
    \rule{\textwidth}{1pt}
  ] % after-code
+

\item
Left-aligned title within two rules (no chapter number):\\
\verb+\titleformat{name=\chapter}[display]
  {\bfseries\Large}
  {}
  {1ex}
  {\titlerule\vspace{2ex}\filright\huge}
  [\vspace{1ex}\titlerule]
+

\item
Title with a bottom rule:\\
\verb+\titleformat{\chapter}
  {\normalfont\Huge\bfseries}
  {\thechapter.}
  {20pt}
  {\Huge}
  [\titlerule]
+

\item
Centered title within two rules (but unnumbered chapters have no rules):\\
\verb+% Adapted from:
% https://tex.stackexchange.com/questions/77788/chapter-titles-using-titlesec
\titleformat{\chapter}[display]
  {\normalfont\bfseries\filcenter}{\LARGE\thechapter}{1ex}
  {\titlerule[2pt]\vspace{2ex}\LARGE}[\vspace{1ex}{\titlerule[2pt]}]

\titleformat{name=\chapter,numberless}[display]
  {\normalfont\LARGE\bfseries\filcenter}{}{1ex}
  {\vspace{2ex}}[\vspace{1ex}]
+

\end{itemize}

You can make this selection in the \verb+tex/config/packages.tex+ file.


\section{Font types}
\label{c2:s:font-types}

You may change the font types in use in the \verb+tex/config/packages.tex+ file.
If you're trying to use different font types make sure you have them installed in your system.


\section{Font styles}
\label{c2:s:font-styles}

\begin{itemize}
\item
\verb+\textnormal{}+ --- \textnormal{document font family}.
\item
\verb+\textrm{}+ --- \textrm{roman font family}.
\item
\verb+\textsf{}+ --- \textsf{sans serif font family}.
\item
\verb+\texttt{}+ --- \texttt{teletypefont family}.
\item
\verb+\textit{}+ --- \textit{italic shape}.
\item
\verb+\textsl{}+ --- \textsl{slanted shape}.
\item
\verb+\textsc{}+ --- \textsc{small capitals}.
\item
\verb+\textbf{}+ --- \textbf{bold}.
\end{itemize}


\section{Colors}
\label{c2:s:colors}

An example of {\color{red} red colored text} from the \texttt{color} package.
And {\color{Blue4} Blue4 colored text} from the \texttt{xcolor} package.

\section{Footnotes}
\label{c2:s:footnotes}

This is a labeled footnote\cref{foot:example}.
A footnote can be referenced multiple times\footnote{\label{foot:example}This is a footnote example.}.
Again, the same footnote is referenced\cref{foot:example}.


\subsection{A table example}
\label{c2:ss:a-table-example}

A table example is shown in \Cref{tab:a-table-example}.

\FloatBarrier
\begin{table}[!htbp]
\caption{A table example.}
\label{tab:a-table-example}
\centering
\begin{tabular}{m{33.3mm}D{33.3mm}F{33.3mm}G{33.3mm}N}
\toprule
justified  & left-aligned & centered   & right-aligned & \\
\midrule
\dummytext & \dummytext   & \dummytext & \dummytext    & \\
\midrule
This is an example & 2 & 3 & 4 & \\
\midrule
Single cell & \multicolumn{3}{|c|}{A multi-column cell} & \\
\midrule
\multirow{2}[1]{*}{A multi-row cell} & A simple & ... & ... & \\
& example & ... & ... & \\
\bottomrule
\end{tabular}
\end{table}
\FloatBarrier



\section{Abbreviations}
\label{c2:s:abbreviations}

\verb+\gls{label}+ and \verb+\glslink{label}{text}+ are two possible commands for making use of abbreviations.
For example, the commands \verb+\gls{afk}+ (first call), \verb+\gls{afk}+ (second call), and \verb+\glslink{afk}{insert specific text}+ produce respectively ``\gls{afk}'', ``\gls{afk}'' and ``\glslink{afk}{insert specific text}''.

A list of some commands follow.

\begin{itemize}
\item
\verb+\gls{afk}+ produces ``\gls{afk}''.
\item
\verb+\glslink{afk}{text}+ produces ``\glslink{afk}{text}''.
\item
\verb+\glsxtrshort{afk}+ and \verb+\as{afk}+ produce ``\glsxtrshort{afk}'' and ``\as{afk}'', respectively.
\item
\verb+\glsxtrlong{afk}+ and \verb+\al{afk}+ produce ``\glsxtrlong{afk}'' and ``\al{afk}'', respectively.
\end{itemize}

Note that the commands \verb+\as{}+ and \verb+\al{}+ are shorter variants.

Other abbreviations include: good work (\as{gw}); have fun (\as{hf}); good work and have fun (\as{gwhf}).
Note that the latter contains nested abbreviations.


\section{Equations}
\label{c2:s:equations}

\Cref{eq:example1} is a numbered equation.

\begin{equation}
\label{eq:example1}
x = 1 + y
\end{equation}

The following equation is not numbered, and thus cannot be referenced.

\begin{equation*}
y = \sum_{i=1}^{N}{x_i}
\end{equation*}

The \verb+\myscore+ pre-defined math function is used in \Cref{eq:example2}.

\begin{equation}
\label{eq:example2}
\myscore(d) = \frac{1}{d^2}
\end{equation}


\section{Figures}
\label{c2:s:figures}

An example of a figure is shown in \Cref{fig:aveiro}. The \verb+\fbox+ command \fbox{draws a box} around its content.

\input{tex/contents/fig/aveiro.tex}


\section{Rotating pages}
\label{c2:s:rotating-pages}

Pages can be displayed horizontally (landscape orientation) using the \verb+landscape+ environment  from the \verb+pdflscape+ package as:

\begin{verbatim}
\begin{landscape}
Add some text or figures.
\end{landscape}
\end{verbatim}


\section{A long table}
\label{c2:s:a-long-table}

\Cref{tab:a-long-table} presents a long table using the \verb+longtable+ package.
This table can span multiple pages.
The \verb+\afterpage+ command forces the table to start at the top of a page.

\afterpage{\input{tex/contents/tab/a-long-table.tex}}\cleardoublepage
