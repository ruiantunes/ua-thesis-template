% Title, Portuguese and English titles, and thesis year.
\newcommand{\authorname}{Rui Marcos\newline Brandão Antunes}
\newcommand{\englishtitle}{\LaTeX\ template for theses at University of Aveiro}
\newcommand{\portuguesetitle}{Template \LaTeX\ para teses na Universidade de Aveiro}
\newcommand{\thesisyear}{2024}

% Removing the lines with \setcounter{page} in the titlepage definition
% to disable page numbering restart.
% https://tex.stackexchange.com/questions/68699/how-to-avoid-page-numbering-being-re-started-by-titlepage
% https://tex.stackexchange.com/questions/27543/what-does-the-titlepage-environment-do-and-what-are-its-benefits
% https://www.tug.org/svn/texlive/trunk/Master/texmf-dist/tex/latex/base/report.cls?view=co
\makeatletter
\if@compatibility
  \renewenvironment{titlepage}
    {%
      \if@twocolumn
        \@restonecoltrue\onecolumn
      \else
        \@restonecolfalse\newpage
      \fi
      \thispagestyle{empty}%
      % \setcounter{page}\z@
    }%
    {\if@restonecol\twocolumn \else \newpage \fi
    }
\else
  \renewenvironment{titlepage}
    {%
      \if@twocolumn
        \@restonecoltrue\onecolumn
      \else
        \@restonecolfalse\newpage
      \fi
      \thispagestyle{empty}%
      % \setcounter{page}\@ne
    }%
    {\if@restonecol\twocolumn \else \newpage \fi
     \if@twoside\else
        % \setcounter{page}\@ne
     \fi
    }
\fi
\makeatother

% First pages are numbered A, B, C, ...
% Also, this avoids wrong back references with the biblatex package.
\pagenumbering{Alph}

\begingroup
% Use Helvetica font in the first pages (according to the UA rules).
% https://tex.stackexchange.com/questions/427245/how-to-use-helvetica-font-in-online-editor
% https://www.overleaf.com/learn/latex/Font_typefaces
% In fact the TeX Gyre Heros is used because it can be used as a
% substitute for Adobe Helvetica.

\ifPDFTeX
\renewcommand{\sfdefault}{qhv}
% \renewcommand{\sfdefault}{phv}
\fi

\ifLuaTeX
% \renewfontfamily\sffamily{Arial}
% \renewfontfamily\sffamily{Arimo}
% \renewfontfamily\sffamily{Helvetica}
\renewfontfamily\sffamily{TeX Gyre Heros}
\fi

% Cover page.
\TitlePage
  \HEADER{\BAR}{\thesisyear}
  \vspace*{14mm}
  \TITLE{\authorname}{\portuguesetitle}
  \vspace*{7mm}
  \TITLE{}{\englishtitle}
\EndTitlePage

% Empty page.
\titlepage\ \endtitlepage

% Initial thesis pages.
\TitlePage
  \HEADER{}{\thesisyear}
  \vspace*{14mm}
  \TITLE{\authorname}{\portuguesetitle}
  \vspace*{7mm}
  \TITLE{}{\englishtitle}
  \vspace*{15mm}
  \TEXT{}{Dissertação/Tese apresentada à Universidade de Aveiro para cumprimento dos requisitos necessários à obtenção do grau de Mestre/Doutor em X, realizada sob a orientação científica de Y, Professor do Departamento Z da Universidade de Aveiro.}
  \vspace*{\fill}
  \TEXT{}{Apoio financeiro da FCT e do FSE no âmbito do III Quadro Comunitário de Apoio. (se aplicável)}
\EndTitlePage

% Empty page.
\titlepage\ \endtitlepage

\TitlePage
  \vspace*{81.5mm}
  \TEXT{\textbf{o júri~/~the jury\newline}}
       {}
  \TEXT{\small presidente~/~president}
       {\textbf{ABC}\newline Professor Catedrático da Universidade de Aveiro}
  \vspace*{5mm}
  \TEXT{\small vogais~/~examiners committee}
       {\textbf{DEF}\newline Professor Catedrático da Universidade de Aveiro (orientador)}
  \vspace*{5mm}
  \TEXT{}
       {\textbf{GHI}\newline Professor Associado da Universidade J (co-orientador)}
  \vspace*{5mm}
  \TEXT{}
       {\textbf{KLM}\newline Professor Catedrático da Universidade N}
\EndTitlePage

% Empty page.
\titlepage\ \endtitlepage

\TitlePage
  \vspace*{81.5mm}
  \TEXT{\textbf{agradecimentos}}
       {... (opcional)}
  \vspace*{12mm}
  \TEXT{\textbf{acknowledgments}}
       {... (optional)}
\EndTitlePage

% Empty page.
\titlepage\ \endtitlepage

\TitlePage
  \vspace*{81.5mm}
  \TEXT{\textbf{palavras-chave}}{Palavra $\cdot$ Chave.}
  \vspace*{12mm}
  \TEXT{\textbf{resumo}}{Este é o primeiro parágrafo do resumo.\newline Segundo parágrafo.\newline Terceiro parágrafo.}
\EndTitlePage

% Empty page.
\titlepage\ \endtitlepage

\TitlePage
  \vspace*{81.5mm}
  \TEXT{\textbf{keywords}}{Key $\cdot$ Word.}
  \vspace*{12mm}
  \TEXT{\textbf{abstract}}{This is the first paragraph of the abstract.\newline Second paragraph.\newline Third paragraph.}
\EndTitlePage

% Empty page.
\titlepage\ \endtitlepage

\TitlePage
  \vspace*{81.5mm}
  \TEXT{\textbf{reconhecimento do uso de ferramentas de IA}}{\textbf{Reconhecimento do uso de tecnologias e ferramentas de Inteligência}\\\textbf{Artificial generativa, programas, e outras ferramentas de apoio.}\\[0.8\baselineskip] Reconheço o uso de [\textit{inserir sistema(s) de IA e respetiva(s) hiperligação(ões)}] para [\textit{indicar utilização específica de IA ou outras tarefas}]. Reconheço a utilização de [\textit{indicar programa, código, ou plataforma}] para [\textit{indicar utilização específica do programa, código, ou plataforma, ou outras tarefas}].\\\\Exemplo 1:\\Reconheço a utilização do ChatGPT 3.5 (OpenAI, \href{https://chatgpt.com}{\texttt{chatgpt.com}}) para resumir as notas iniciais e para rever o rascunho final.\\\\Exemplo 2:\\Não foram utilizados no presente documento quaisquer conteúdos gerados por tecnologias de IA.}
  \vspace*{12mm}
  \TEXT{\textbf{acknowledgment of use of AI tools}}{\textbf{Recognition of the use of generative Artificial Intelligence technologies}\\\textbf{and tools, software, and other support tools.}\\[0.8\baselineskip] I acknowledge the use of [\textit{insert AI system(s) and link(s)}] to [\textit{specific use of generative artificial intelligence or other tasks}]. I acknowledge the use of [\textit{software, code, or platform}] to [\textit{specific use of sofware, code, or platform, or other tasks}].\\\\Example 1:\\I acknowledge the use of ChatGPT 3.5 (OpenAI, \href{https://chatgpt.com}{\texttt{chatgpt.com}}) to summarize the initial notes and to proofread the final draft.\\\\Example 2:\\No content generated by AI technologies has been used in this document.}
\EndTitlePage

% Empty page.
\titlepage\ \endtitlepage

% End of Helvetica font.
\endgroup

% Specifying header content. In this case it only shows chapter
% information: left position at even pages, right position at odd pages.
\setlength\headheight{16pt}
\pagestyle{fancy}
\fancyhf{}
\fancyhead[LO,RE]{\fontsize{12}{14.4}}
\fancyhead[LE,RO]{\fontsize{12}{14.4}\textsc{\nouppercase{\leftmark}}}

% To change the font size of the page numbering in all pages.
\fancyfoot[C]{\small\thepage}

% From the "fancyhdr" package documentation.
% "Some LATEX commands, like \chapter, use the \thispagestyle command to
% automatically switch to the plain page style, thus ignoring the page
% style currently in effect."

% The "fancyhdr" packages does not apply same header/footer on chapter
% and non-chapter pages.
% https://tex.stackexchange.com/questions/117328/fancyhdr-does-not-apply-same-header-footer-on-chapter-and-non-chapter-pages

\fancypagestyle{plain}{%
  % Clear all header and footer fields.
  \fancyhf{}
  % Except the center.
  \fancyfoot[C]{\small\thepage}
  \renewcommand{\headrulewidth}{0pt}
  \renewcommand{\footrulewidth}{0pt}
}

% To specify 1x or 1.5x vertical spacing between lines.
% \singlespacing
\onehalfspacing

% Tables of contents, of figures, of tables.

% To count the following pages with roman numbering.
\pagenumbering{roman}

\tableofcontents
\cleardoublepage

\listoffigures
\cleardoublepage

\listoftables
\cleardoublepage

\begingroup
% To locally reduce vertical space between entries.
\setlist{itemsep=0pt,topsep=0pt,parsep=0pt,partopsep=0pt}
\printnoidxglossary
\endgroup
\cleardoublepage

% To specify 1x or 1.5x vertical spacing between lines.
% \singlespacing
\onehalfspacing

% The chapters.

% To count the following pages with Arabic numerals.
\pagenumbering{arabic}
