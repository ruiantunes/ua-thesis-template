% Select the language (correct expressions).
\makeatletter
\ifportuguese@use
% "cleveref" Portuguese names.
\crefname{equation}{equação}{equações}
\Crefname{equation}{Equação}{Equações}
\crefname{table}{tabela}{tabelas}
\Crefname{table}{Tabela}{Tabelas}
\crefname{figure}{figura}{figuras}
\Crefname{figure}{Figura}{Figuras}
\crefname{chapter}{capítulo}{capítulos}
\Crefname{chapter}{Capítulo}{Capítulos}
\crefname{section}{secção}{secções}
\Crefname{section}{Secção}{Secções}

\crefname{appendix}{apêndice}{apêndices}
\Crefname{appendix}{Apêndice}{Apêndices}

\crefname{annex}{anexo}{anexos}
\Crefname{annex}{Anexo}{Anexos}

\else
\ifukenglish@use
% "cleveref" English names.
\crefname{appendix}{appendix}{appendices}
\Crefname{appendix}{Appendix}{Appendices}

\crefname{annex}{annex}{annexes}
\Crefname{annex}{Annex}{Annexes}

\else
\ifusenglish@use
% "cleveref" English names.
\crefname{appendix}{appendix}{appendices}
\Crefname{appendix}{Appendix}{Appendices}

\crefname{annex}{annex}{annexes}
\Crefname{annex}{Annex}{Annexes}

\fi
\fi
\fi
\makeatother

% Title and author. This is not mandatory to change, because this is
% only used by the "\maketitle" command which it is not used in this
% template.
\title{The title}
\author{The author}

% The counter "tocdepth" controls the "Table of contents" depth.
%   Depth of the table of contents (TOC):
%     1 ... chapter and sections;
%     2 ... chapters, sections, and subsections;
%     3 ... chapters, sections, subsections, and subsubsections.
\newcommand{\mytocdepth}{2}
\setcounter{tocdepth}{\mytocdepth}

% The counter "secnumdepth" controls which section levels are numbered.
% https://latexref.xyz/Sectioning.html
\newcommand{\mysecnumdepth}{2}
\setcounter{secnumdepth}{\mysecnumdepth}

% List of abbreviations.
\input{tex/config/abbreviations.tex}

% Math functions declaration.
% https://tex.stackexchange.com/questions/95171/how-can-i-define-user-defined-functions-in-math-mode
% https://tex.stackexchange.com/questions/130510/write-text-correctly-in-equations
\DeclareMathOperator{\myscore}{score}

% List of manually hyphenated words in case hyphenation is enabled
% (check the "hyphenat" package). Hyphenation in the middle is desired
% (better readability).
\hyphenation{
% English words.
ex-am-ple
mas-sa-chu-setts
test
% Portuguese words.
cien-tí-fico
con-cei-tos
exem-plo
infor-mações
tes-te
}
