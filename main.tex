%
% Copyright 1999, 2007, 2009 Tomás Oliveira e Silva
% Copyright 2018, 2019, 2020, 2021, 2022, 2024, 2025 Rui Antunes
%
% LaTeX template for theses at University of Aveiro by Rui Antunes.
% https://github.com/ruiantunes/ua-thesis-template
%
% This code is based on the original LaTeX template created by
% professor Tomás Oliveira e Silva.
% http://sweet.ua.pt/tos/TeX.html
%

% For a faster compilation (debugging purposes only), using image
% placeholders, employ the "draft" option in the "graphicx" package:
% \PassOptionsToPackage{draft}{graphicx}

% Font size possible values are restricted to 10pt, 11pt and 12pt.
\documentclass[%
11pt,%
a4paper,%
twoside,%
openright,%
]{report}

% Select language (Portuguese, UK English, or US English).
% % Portuguese names.
\renewcommand*{\contentsname}{Índice}
\renewcommand*{\listfigurename}{Lista de figuras}
\renewcommand*{\listtablename}{Lista de tabelas}
\renewcommand*{\glossaryname}{Lista de abreviações}
\renewcommand*{\bibname}{Referências}
\newcommand*{\notationname}{Notação}

% "cleveref" Portuguese names.
\crefname{equation}{equação}{equações}
\Crefname{equation}{Equação}{Equações}
\crefname{table}{tabela}{tabelas}
\Crefname{table}{Tabela}{Tabelas}
\crefname{figure}{figura}{figuras}
\Crefname{figure}{Figura}{Figuras}
\crefname{chapter}{capítulo}{capítulos}
\Crefname{chapter}{Capítulo}{Capítulos}
\crefname{section}{secção}{secções}
\Crefname{section}{Secção}{Secções}

% Changing conjunctions. See "cleveref" documentation.
% https://ctan.org/pkg/cleveref
% https://tex.stackexchange.com/questions/275828/cleveref-changing-conjunction-for-pair
\renewcommand*{\crefpairconjunction}{\ e\ }
\renewcommand*{\crefmiddleconjunction}{,\ }
\renewcommand*{\creflastconjunction}{, e\ }
\renewcommand*{\crefrangeconjunction}{\ a\ }

\renewcommand*{\appendixname}{Apêndice}
\renewcommand*{\appendixtocname}{Apêndices}
\renewcommand*{\appendixpagename}{Apêndices}

\crefname{appendix}{apêndice}{apêndices}
\Crefname{appendix}{Apêndice}{Apêndices}

% \input{tex/config/language/ukenglish.tex}
\input{tex/config/language/usenglish.tex}

% UA thesis style file.
% Choose one of the following scientific areas:
%   - accounting
%   - arts
%   - arts+humanities
%   - economy
%   - education
%   - engineering
%   - health
%   - humanities
%   - misc
%   - sciences
% Remove the "draft" option for the final document.
\usepackage[%
draft,%
engineering,%
% MAP,%
% oldLogo,%
% horizontalRule,%
]{uathesis}

% Additional packages.
%
% Note that the order that the packages are imported sometimes it
% matters. That is, some packages should be imported before others.
% Otherwise, it may not work correctly.
%
% You may have to add new packages, remove packages, or modify their
% input arguments to obtain the intended functionality.
%

% This package identifies which TeX engine is in use (pdfTeX or LuaTeX).
\usepackage{iftex}

\ifPDFTeX
% Input encoding.
% LuaTeX uses UTF-8 encoding by default.
\usepackage[utf8]{inputenc}
% Font encoding.
\usepackage[T1]{fontenc}
\fi

\usepackage{textcomp}

% Using "Latin Modern" font is better.
% https://tex.stackexchange.com/questions/88368/how-do-i-invoke-cm-super
% http://tug.org/pipermail/macostex-archives/2004-July/006964.html
% https://tex.stackexchange.com/questions/30814/identity-of-latex-default-sans-serif-font/30816
% https://tex.stackexchange.com/questions/66674/when-i-load-multiple-font-packages-which-one-will-win
% https://tex.stackexchange.com/questions/25249/how-do-i-use-a-particular-font-for-a-small-section-of-text-in-my-document
\usepackage{lmodern}

% Select the language.
% The "csquotes" package should be used with the "babel" package.
% https://tex.stackexchange.com/questions/229638/package-biblatex-warning-babel-polyglossia-detected-but-csquotes-missing
\makeatletter
\ifportuguese@use
\usepackage[portuguese]{babel}
\usepackage[style=portuguese,threshold=1]{csquotes}
\else
\ifukenglish@use
\usepackage[UKenglish]{babel}
\usepackage[style=UKenglish,threshold=1]{csquotes}
\else
\ifusenglish@use
\usepackage[USenglish]{babel}
\usepackage[style=USenglish,threshold=1]{csquotes}
\fi
\fi
\fi
\makeatother

\usepackage[%
a4paper,%
% These margins are according to the UA rules.
% http://www.ua.pt/sga/PageText.aspx?id=4630
left=30mm,%
right=25mm,%
bottom=30mm,%
top=30mm,%
% However, one may want to slightly increase the right margin, so there
% is no shift when changing pages (using single page view) in a PDF
% document viewer.
% https://en.wikibooks.org/wiki/Basic_Book_Design/Margins
% left=30mm,%
% right=30mm,%
% bottom=30mm,%
% top=30mm,%
% To show visible frames for the text area and page.
% showframe,%
]{geometry}

% Mathematical utilities.
\usepackage{amsmath}
\usepackage{amssymb}

\ifPDFTeX
% Access to bold math symbols.
% \usepackage{bm}

% Font type selection in pdfTeX.

% Linux libertine font type.
% https://ctan.org/pkg/libertine
% \usepackage{libertine}
% \renewcommand{\ttdefault}{lmtt}
% \usepackage[libertine]{newtxmath}

\else
\ifLuaTeX

% Font type selection in LuaTeX.
% https://ctan.org/pkg/fontspec
\usepackage{fontspec}

% By default it is used the CMU font:
%   main: CMU Serif
%   sans: CMU Sans Serif
%   mono: CMU Typewriter Text
%   math: Latin Modern Math
%
% Libertinus font is also a nice choice.
% https://github.com/alerque/libertinus
% https://en.wikipedia.org/wiki/Linux_Libertine
% If you want to use Libertinus select:
%   main: Libertinus Serif
%   sans: Libertinus Sans
%   mono: Inconsolata (one monospaced font at your choice)
%   math: Libertinus Math
%
% You may need to install the fonts if they are missing. You can use
% other fonts that are not mentioned here.
% https://www.overleaf.com/learn/latex/Questions/Which_OTF_or_TTF_fonts_are_supported_via_fontspec%3F
%
% \setmainfont{BaskervilleF}
% \setmainfont{CMU Concrete}
\setmainfont{CMU Serif}
% \setmainfont{FreeSerif}
% \setmainfont{Georgia}\renewcommand{\textsc}{\textrm}
% \setmainfont{Libertinus Serif}
% \setmainfont{Linux Libertine O}
% \setmainfont{Noto Serif}
% \setmainfont{Quattrocento}\renewcommand{\textsc}{\textrm}
% \setmainfont{TeX Gyre Bonum}
% \setmainfont{TeX Gyre Pagella}
% \setmainfont{TeX Gyre Schola}
% \setmainfont{TeX Gyre Termes}
% \setmainfont{Times New Roman}\renewcommand{\textsc}{\textrm}
%
% \setsansfont{Arial}
% \setsansfont{Arimo}
% \setsansfont{CMU Bright}
\setsansfont{CMU Sans Serif}
% \setsansfont{CMU Sans Serif Demi Condensed}\renewcommand{\textsf}{\textnormal}
% \setsansfont{Helvetica}
% \setsansfont{Libertinus Sans}
% \setsansfont{Linux Biolinum O}
% \setsansfont{TeX Gyre Heros}
% \setsansfont{Verdana}
%
% \setmonofont{Courier New}
% \setmonofont{Courier Prime}
\setmonofont{CMU Typewriter Text}
% \setmonofont{Inconsolata}
% \setmonofont{Inconsolata}[Scale=0.9]
% \setmonofont{Inconsolata}[Scale=MatchLowercase]
% \setmonofont{Inconsolata}[Scale=MatchUppercase]
% \setmonofont{JetBrains Mono}
% \setmonofont{Latin Modern Mono}
% \setmonofont{Libertinus Mono}
% \setmonofont{Lucida Console}
% \setmonofont{Menlo}
% \setmonofont{Monaco}
% \setmonofont{Ubuntu Mono}
% \setmonofont{Roboto Mono}
% \setmonofont{Source Code Pro}
% \setmonofont{Space Mono}
%
% To use \mathbf{} without errors.
% https://github.com/wspr/unicode-math/issues/556
% https://tex.stackexchange.com/questions/528831/why-doesnt-the-bm-package-work-with-the-unicode-math-package
\usepackage{unicode-math}
\setmathfont{Latin Modern Math}
% \setmathfont{Libertinus Math}

\else
\RequireLuaTeX
\fi
\fi

% Improve line breaking mechanisms.
% https://tex.stackexchange.com/questions/226286/how-to-deal-with-many-overfull-hboxes-in-large-document/239955#239955
% https://tex.stackexchange.com/questions/162137/loading-microtype-before-or-after-the-font
% https://tex.stackexchange.com/questions/241343/what-is-the-meaning-of-fussy-sloppy-emergencystretch-tolerance-hbadness
% https://texfaq.org/FAQ-overfull
% https://tex.stackexchange.com/questions/636667/why-doesnt-cleveref-work-in-portuguese/636675#636675
\usepackage{microtype}

% For customizing chapter styles.
% https://texblog.org/2012/07/03/fancy-latex-chapter-styles/
% Warning: the "titlesec" package may conflict with the "hyperref"
% package. The package "titlesec" should be loaded before "hyperref".
% https://tex.stackexchange.com/questions/364010/hyperref-and-titlesec-conflict-and-warning
\usepackage{titlesec}

% Select the preferred chapter style.
% % Adapted from:
% https://tug.ctan.org/macros/latex/contrib/titlesec/titlesec.pdf#subsection.8.2

\titleformat{\chapter}[display]
  {\normalfont\huge\bfseries}{\chaptertitlename\ \thechapter}{20pt}{\Huge}

\titlespacing*{\chapter}{0pt}{50pt}{40pt}

% Veelo chapter style. Adapted from:
% https://tex.stackexchange.com/questions/487917/setting-chapter-style-with-bar
% https://tex.stackexchange.com/questions/177144/chapter-style-report-class
% Note that you can change \rule{30pt} to another width (such as 28pt
% or 32pt if you prefer a thinner or thicker bar).
\newcommand*\chapnamefont{\normalfont\LARGE\MakeUppercase}
\newcommand*\chapnumfont{\normalfont\Huge}
\newcommand*\chaptitlefont{\normalfont\Huge\bfseries}

\newlength\beforechapskip
\newlength\midchapskip
\setlength\midchapskip{\paperwidth}
\addtolength\midchapskip{-\textwidth}
\addtolength\midchapskip{-\oddsidemargin}
\addtolength\midchapskip{-1in}
% Also adjust this value accordingly to your preference.
\setlength\beforechapskip{17mm}

\titleformat{\chapter}[display]
  {\normalfont\filleft}
  {{\chapnamefont\chaptertitlename}%
    \rlap{\hspace{.8em}%
      \makebox[\dimexpr\oddsidemargin+\hoffset+1in][s]{{\resizebox{!}{\beforechapskip}{\chapnumfont\thechapter}}\hfill%
      \rule{30pt}{\beforechapskip}}%
    }%
  }%
  {25pt}
  {\chaptitlefont}
\titlespacing*{\chapter}
  {0pt}{40pt}{40pt}

% % Adapted from:
% https://pt.overleaf.com/learn/latex/Sections_and_chapters
\titleformat
  {\chapter} % command
  [display] % shape
  {\bfseries\LARGE} % format
  {\chaptertitlename\ \thechapter} % label
  {0.5ex} % sep
  {
    \rule{\textwidth}{1pt}
    \vspace{1ex}
    \centering
    \huge
  } % before-code
  [
    \vspace{-2.0ex}%
    \rule{\textwidth}{1pt}
  ] % after-code

% \titleformat{name=\chapter}[display]
  {\bfseries\Large}
  {}
  {1ex}
  {\titlerule\vspace{2ex}\filright\huge}
  [\vspace{1ex}\titlerule]

% \titleformat{\chapter}
  {\normalfont\Huge\bfseries}
  {\thechapter.}
  {20pt}
  {\Huge}
  [\titlerule]

% % Adapted from:
% https://tex.stackexchange.com/questions/77788/chapter-titles-using-titlesec
\titleformat{\chapter}[display]
  {\normalfont\bfseries\filcenter}{\LARGE\thechapter}{1ex}
  {\titlerule[2pt]\vspace{2ex}\LARGE}[\vspace{1ex}{\titlerule[2pt]}]

\titleformat{name=\chapter,numberless}[display]
  {\normalfont\LARGE\bfseries\filcenter}{}{1ex}
  {\vspace{2ex}}[\vspace{1ex}]


% Line spacing. The "setspace" package has to be loaded before the
% "hyperref" package to ensure that the links of footnotes are correct.
% https://tex.stackexchange.com/questions/203439/footnotes-misbehaving-in-report-go-to-front-page-but-behave-correctly-in-other-r
\usepackage{setspace}

\usepackage[font=onehalfspacing]{caption}

% To use more colors.
\usepackage[x11names,table]{xcolor}

% To rotate pages.
\usepackage{pdflscape}

% Managing URLs.
\usepackage[hyphens,spaces]{xurl}

% International System of Units (SI).
\usepackage{siunitx}

% BibLaTeX bibliography. Select the preferred citation style.
% Alternatively, you can configure your own.
% From the "biblatex" documentation: "When using the hyperref package,
% it is preferable to load it after biblatex."
% \input{tex/config/biblatex/authoryear.tex}
\input{tex/config/biblatex/authoryear-comp.tex}
% % BibLaTeX numeric style.
\usepackage[%
backend=biber,%
% Print back references in the bibliography.
% backref=false,%
backref=true,%
bibstyle=ieee,%
citestyle=numeric-comp,%
dashed=false,%
defernumbers=true,%
hyperref=true,%
maxbibnames=999,%
minbibnames=1,%
maxcitenames=2,%
mincitenames=1,%
sorting=none,%
sortcites=false,%
]{biblatex}

% \input{tex/config/biblatex/numeric.tex}

% Input ".bib" file name with the references.
\bibliography{refs}

% To break URLs. These values are recommended in the "xurl" package.
\setcounter{biburllcpenalty}{100}
\setcounter{biburlucpenalty}{200}
\setcounter{biburlnumpenalty}{100}

% Add a line break before the URL.
% https://tex.stackexchange.com/questions/27953/linebreak-before-url-with-biblatex-alphabetic
\DeclareFieldFormat{formaturl}{\iffieldundef{url}{}{\newline #1}}
\renewbibmacro*{url+urldate}{%
  \printtext[formaturl]{%
    \printfield{url}}%
  \iffieldundef{urlyear}
    {}
    {\setunit*{\addspace}%
     \printtext[urldate]{\printurldate}}}

% Changing master's thesis text.
% https://tex.stackexchange.com/questions/226989/biblatex-changing-key-mastersthesis-to-mphilthesis
\DefineBibliographyStrings{portuguese}{%
  mathesis = {Dissertação de mestrado},
}
\DefineBibliographyStrings{english}{%
  % mathesis = {MSc thesis},
  mathesis = {Master's thesis},
}

% Appendices.
% https://ctan.org/pkg/appendix
% https://texfaq.org/FAQ-appendix
% https://tex.stackexchange.com/questions/49643/making-appendix-for-thesis
% Load the "appendix" package before the "hyperref" package to avoid
% conflicts.
% Do not use the "page" option if you do not want to create an
% additional page with only the title "Appendices".
\usepackage[toc,page]{appendix}
% \usepackage[toc]{appendix}

% To restore the counter at the end of the "appendices" environment.
\renewcommand{\restoreapp}{}

% References (citations, images, tables, among others).
% Links cannot break between pages (it is mandatory to manually
% rearrange the text).
% https://tex.stackexchange.com/questions/1522/pdfendlink-ended-up-in-different-nesting-level-than-pdfstartlink
\usepackage[%
breaklinks,%
% draft,%
hidelinks,%
colorlinks,%
linktoc=all,%
% linktoc=page,%
% For debugging.
% allcolors=blue,%
% For final printed version. Choose the colors you want.
allcolors=black,%
linkcolor=black,%
citecolor=blue,%
urlcolor=blue,%
]{hyperref}

% To make the hyperlink put the cursor at the beginning of the figure
% (that is, not in the figure caption).
\usepackage[all]{hypcap}

% The "cleveref" package has to be loaded after the "hyperref" package.
% The "cleveref" has to be loaded after the "amsmath" package to
% correctly reference equations.
% https://tex.stackexchange.com/questions/148699/equation-reference-undefined-when-using-cref-and-packageamsmath
\usepackage[nameinlink,noabbrev]{cleveref}

% Multiple references to the same footnote.
% https://tex.stackexchange.com/questions/10102/multiple-references-to-the-same-footnote-with-hyperref-support-is-there-a-bett
\crefformat{footnote}{#2\footnotemark[#1]#3}

% To create hyperlinks from the footnotes at the bottom of the page,
% back to the occurrence of the footnote in the main text.
% This package conflicts with the "\footcitetext" command from the
% "biblatex" package.
\usepackage{footnotebackref}

% Table footnote.
% https://tex.stackexchange.com/questions/1583/footnotes-in-tables
% From the "tablefootnote" documentation:
% When the document is compiled with LuaTeX, hyperlinks in rotated
% content will be misplaced. Use it with caution.
% \usepackage{tablefootnote}

% Header and footer treatment.
\usepackage{fancyhdr}

% Specifying header content. In this case it only shows chapter
% information: left position at even pages, right position at odd pages.
\setlength\headheight{16pt}
\pagestyle{fancy}
\fancyhf{}
\fancyhead[LO,RE]{\fontsize{12}{14.4}}
\fancyhead[LE,RO]{\fontsize{12}{14.4}\textsc{\nouppercase{\leftmark}}}

% To change the font size of the page numbering in all pages.
\fancyfoot[C]{\small\thepage}

% From the "fancyhdr" package documentation.
% "Some LATEX commands, like \chapter, use the \thispagestyle command to
% automatically switch to the plain page style, thus ignoring the page
% style currently in effect."

% The "fancyhdr" packages does not apply same header/footer on chapter
% and non-chapter pages.
% https://tex.stackexchange.com/questions/117328/fancyhdr-does-not-apply-same-header-footer-on-chapter-and-non-chapter-pages

\fancypagestyle{plain}{%
  % Clear all header and footer fields.
  \fancyhf{}
  % Except the center.
  \fancyfoot[C]{\small\thepage}
  \renewcommand{\headrulewidth}{0pt}
  \renewcommand{\footrulewidth}{0pt}
}

% To place floats (figures, tables) at fixed position using the
% "\FloatBarrier" command.
\usepackage{placeins}

% To allow tables spanning multiple pages.
% https://texfaq.org/FAQ-longtab
% To prevent longtable breaking in multirow.
% https://tex.stackexchange.com/questions/64956/prevent-longtable-breaking-in-multirow
% A longtable example.
% http://users.sdsc.edu/~ssmallen/latex/longtable.html
\usepackage{longtable}

% To enhance the quality of tables.
\usepackage{booktabs}

% To specify vertical alignment in rows (tables).
\usepackage{array}

% Horizontally centered and right aligned.
% https://tex.stackexchange.com/questions/154958/vertically-centered-and-right-left-center-horizontal-alignment-in-tabular

% Note: the following shortcuts for column alignments are easy to
% memorize. They form a 3x3 square in the QWERTY keyboard layout.

% --- Top (vertical alignment) ---
% Justified.
% p{...}
% Left-aligned.
\newcolumntype{E}[1]{>{\raggedright\arraybackslash}p{#1}}
% Centered.
\newcolumntype{R}[1]{>{\centering\arraybackslash}p{#1}}
% Right-aligned.
\newcolumntype{T}[1]{>{\raggedleft\arraybackslash}p{#1}}

% --- Middle (vertical alignment) ---
% Justified.
% m{...}
% Left-aligned.
\newcolumntype{D}[1]{>{\raggedright\arraybackslash}m{#1}}
% Centered.
\newcolumntype{F}[1]{>{\centering\arraybackslash}m{#1}}
% Right-aligned.
\newcolumntype{G}[1]{>{\raggedleft\arraybackslash}m{#1}}

% --- Bottom (vertical alignment) ---
% Justified.
% b{...}
% Left-aligned.
\newcolumntype{C}[1]{>{\raggedright\arraybackslash}b{#1}}
% Centered.
\newcolumntype{V}[1]{>{\centering\arraybackslash}b{#1}}
% Right-aligned.
\newcolumntype{B}[1]{>{\raggedleft\arraybackslash}b{#1}}

% Extra column with zero width and no padding. This is used to avoid the
% incorrect alignment in the last column.
% https://tex.stackexchange.com/questions/159257/increase-latex-table-row-height
% https://tex.stackexchange.com/questions/68732/vertical-alignment-in-table-m-column-row-size-problem-in-last-column
\newcolumntype{N}{@{}m{0pt}@{}}

% Empty column separator.
% https://texfaq.org/FAQ-fixwidtab
\newcolumntype{O}{@{}}

% To use 2% of the \textwidth as column separator.
\newcolumntype{I}{@{\hskip0.02\textwidth}}

% Multirow in tables.
\usepackage{multirow}

% New lines in multirow cells.
% https://tex.stackexchange.com/questions/331716/newline-in-multirow-environment
\usepackage{makecell}

% To change the itemize character.
\usepackage{enumitem}

% To compact the "itemize" list.
% https://stackoverflow.com/questions/4968557/latex-very-compact-itemize/4974583#4974583
\setlist{itemsep=5pt,topsep=5pt,parsep=5pt,partopsep=5pt}

% To place "longtable" tables at the top of the page.
\usepackage{afterpage}

% Hyphenation throughout the document.
\usepackage{hyphenat}
% To disable hyphenation use the "none" option.
% \usepackage[none]{hyphenat}

% For rotating text with the "turn" command.
\usepackage{rotating}

% To use the "\sfrac" command.
\usepackage{xfrac}

% Disable displaying page number, header and footer on empty pages.
\usepackage{emptypage}

% Colored boxes.
% \usepackage[most]{tcolorbox}

% To force not to break the line after a hyphen. This conflicts with
% latexdiff. Use it with caution.
% https://tex.stackexchange.com/questions/103608/how-to-force-latex-not-to-break-the-line-after-a-hyphen
% \usepackage[shortcuts]{extdash}

% For creating a list of abbreviations.
\usepackage[%
nogroupskip=false,%
nonumberlist=true,%
nopostdot=true,%
% style=alttree,%
style=listdotted,%
% style=long,%
% style=super,%
shortcuts=abbreviations,%
% The "uathesis" style file is already adding the glossary into the
% table of contents.
toc=false,%
]{glossaries-extra}

% To revert \markright, \markboth and \@starttoc related with the
% \MakeUpperCase command. This is required to not mix up with the
% uppercase of chapter headings.
\glsxtrRevertMarks

% To disable all abbreviation hyperlinks.
% https://tex.stackexchange.com/questions/25805/disable-hyperlinks-in-some-entries-for-glossaries
% \glsdisablehyper

% To add links inside captions.
% https://tex.stackexchange.com/questions/30800/gls-inside-caption-without-protect
\robustify{\gls}

% To specify the widest name of the short form of an abbreviation.
% This applies when the "alttree" style is used.
\glssetwidest{123456789012345}

% To specify dotted distance between abbreviation and description.
% This applies when the "listdotted" style is used.
% https://tex.stackexchange.com/questions/17956/dotted-distance-between-acronym-and-description-in-acronymlist-from-glossaries
\setlength{\glslistdottedwidth}{.28\textwidth}

% To remove bold of the short forms in the glossary.
\renewcommand{\glsnamefont}[1]{\normalfont #1}

% To generate the glossary.
% https://tex.stackexchange.com/questions/243750/acronym-glossary-doesnt-appear
\makenoidxglossaries


% Before begin document.
% Title and author. This is not mandatory to change, because this is
% only used by the "\maketitle" command which it is not used in this
% template.
\title{The title}
\author{The author}

% The counter "tocdepth" controls the "Table of contents" depth.
%   Depth of the table of contents (TOC):
%     1 ... chapter and sections;
%     2 ... chapters, sections, and subsections;
%     3 ... chapters, sections, subsections, and subsubsections.
\newcommand{\mytocdepth}{2}
\setcounter{tocdepth}{\mytocdepth}

% The counter "secnumdepth" controls which section levels are numbered.
% https://latexref.xyz/Sectioning.html
\newcommand{\mysecnumdepth}{2}
\setcounter{secnumdepth}{\mysecnumdepth}

% List of abbreviations.
\input{tex/config/abbreviations.tex}

% Math functions declaration.
% https://tex.stackexchange.com/questions/95171/how-can-i-define-user-defined-functions-in-math-mode
% https://tex.stackexchange.com/questions/130510/write-text-correctly-in-equations
\DeclareMathOperator{\myscore}{score}

% List of manually hyphenated words in case hyphenation is enabled
% (check the "hyphenat" package). Hyphenation in the middle is desired
% (better readability).
\hyphenation{
% English words.
ex-am-ple
mas-sa-chu-setts
test
% Portuguese words.
cien-tí-fico
con-cei-tos
exem-plo
infor-mações
tes-te
}


\begin{document}

% After begin document.
% Select the language (correct expressions).
\makeatletter
\ifportuguese@use
% Portuguese names.
\renewcommand*{\contentsname}{Índice}
\renewcommand*{\listfigurename}{Lista de figuras}
\renewcommand*{\listtablename}{Lista de tabelas}
\renewcommand*{\glossaryname}{Lista de abreviações}
\renewcommand*{\bibname}{Referências}
\newcommand*{\notationname}{Notação}

% "cleveref" Portuguese names.
\crefname{equation}{equação}{equações}
\Crefname{equation}{Equação}{Equações}
\crefname{table}{tabela}{tabelas}
\Crefname{table}{Tabela}{Tabelas}
\crefname{figure}{figura}{figuras}
\Crefname{figure}{Figura}{Figuras}
\crefname{chapter}{capítulo}{capítulos}
\Crefname{chapter}{Capítulo}{Capítulos}
\crefname{section}{secção}{secções}
\Crefname{section}{Secção}{Secções}

% Changing conjunctions. See "cleveref" documentation.
% https://ctan.org/pkg/cleveref
% https://tex.stackexchange.com/questions/275828/cleveref-changing-conjunction-for-pair
\renewcommand*{\crefpairconjunction}{\ e\ }
\renewcommand*{\crefmiddleconjunction}{,\ }
\renewcommand*{\creflastconjunction}{, e\ }
\renewcommand*{\crefrangeconjunction}{\ a\ }

\renewcommand*{\appendixname}{Apêndice}
\renewcommand*{\appendixtocname}{Apêndices}
\renewcommand*{\appendixpagename}{Apêndices}

\crefname{appendix}{apêndice}{apêndices}
\Crefname{appendix}{Apêndice}{Apêndices}

\else
\ifukenglish@use
% English alternative names.
\renewcommand*{\contentsname}{Table of contents}
\renewcommand*{\listfigurename}{List of figures}
\renewcommand*{\listtablename}{List of tables}
\renewcommand*{\glossaryname}{List of abbreviations}
\renewcommand*{\bibname}{References}
\newcommand*{\notationname}{Notation}

\crefname{appendix}{appendix}{appendices}
\Crefname{appendix}{Appendix}{Appendices}

\else
\ifusenglish@use
% English alternative names.
\renewcommand*{\contentsname}{Table of contents}
\renewcommand*{\listfigurename}{List of figures}
\renewcommand*{\listtablename}{List of tables}
\renewcommand*{\glossaryname}{List of abbreviations}
\renewcommand*{\bibname}{References}
\newcommand*{\notationname}{Notation}

\crefname{appendix}{appendix}{appendices}
\Crefname{appendix}{Appendix}{Appendices}

\fi
\fi
\fi
\makeatother

% To change symbol of itemize (size of bullet).
\renewcommand*{\labelitemi}{$\vcenter{\hbox{\small$\bullet$}}$}

% Horizontal line to separate floats (figures and tables) from text.
% \def\topfigrule{\kern 7.8pt \hrule width\textwidth\kern -8.2pt\relax}
% \def\dblfigrule{\kern 7.8pt \hrule width\textwidth\kern -8.2pt\relax}
% \def\botfigrule{\kern -7.8pt \hrule width\textwidth\kern 8.2pt\relax}

% To increase vertical line spacing in tables.
\renewcommand{\arraystretch}{1.2}

% To change column separation in tables.
% \setlength\tabcolsep{10pt}
\setlength\tabcolsep{0.02\textwidth}

% New commands.

% https://tex.stackexchange.com/questions/32160/new-line-after-paragraph
\newcommand{\myparagraph}[1]{\paragraph{#1}\mbox{}}
\newcommand{\mysubparagraph}[1]{\subparagraph{#1}\mbox{}}

\makeatletter
\newcommand\myfontsize{\f@size pt}
\makeatother

\newcommand{\myrule}{\noindent\rule{\textwidth}{1pt}}

% \newcommand{\todo}{{\color{red}\bfseries...to\ do...}}

\newcommand{\dummytext}{a bc def ghij klm no p qr stu vwxy z}

% Temporary width.
% https://texfaq.org/FAQ-findwidth
% \newlength{\mywidth}

% Latin expressions.
% https://en.wikipedia.org/wiki/List_of_Latin_phrases
\newcommand{\adhoc}{\emph{ad hoc}}
\newcommand{\aposteriori}{\emph{a posteriori}}
\newcommand{\apriori}{\emph{a priori}}
\newcommand{\cf}{\emph{cf.}}
\newcommand{\defacto}{\emph{de facto}}
\newcommand{\eg}{\emph{e.g.}}
\newcommand{\ergo}{\emph{ergo}}
\newcommand{\etal}{\emph{et al.}}
\newcommand{\etc}{\emph{etc}}
\newcommand{\ie}{\emph{i.e.}}
\newcommand{\re}{\emph{re}}
\newcommand{\versus}{\emph{versus}}
\newcommand{\viceversa}{\emph{vice versa}}
\newcommand{\viz}{\emph{viz.}}
\newcommand{\vs}{\emph{vs}}

% To make text never go over the right margin.
% https://tex.stackexchange.com/questions/9107/how-can-i-make-my-text-never-go-over-the-right-margin-by-always-hyphenating-or-b
\sloppy

% Disable vertical stretch across the pages.
% https://tex.stackexchange.com/questions/10743/why-does-latex-stretch-small-sections-across-the-whole-page-vertically
\raggedbottom


% First pages (can be commented to reduce compilation time).
% Title, Portuguese and English titles, and thesis year.
\newcommand{\authorname}{Rui Marcos\newline Brandão Antunes}
\newcommand{\englishtitle}{\LaTeX\ template for theses at University of Aveiro}
\newcommand{\portuguesetitle}{Template \LaTeX\ para teses na Universidade de Aveiro}
\newcommand{\thesisyear}{2024}

% Removing the lines with \setcounter{page} in the titlepage definition
% to disable page numbering restart.
% https://tex.stackexchange.com/questions/68699/how-to-avoid-page-numbering-being-re-started-by-titlepage
% https://tex.stackexchange.com/questions/27543/what-does-the-titlepage-environment-do-and-what-are-its-benefits
% https://www.tug.org/svn/texlive/trunk/Master/texmf-dist/tex/latex/base/report.cls?view=co
\makeatletter
\if@compatibility
  \renewenvironment{titlepage}
    {%
      \if@twocolumn
        \@restonecoltrue\onecolumn
      \else
        \@restonecolfalse\newpage
      \fi
      \thispagestyle{empty}%
      % \setcounter{page}\z@
    }%
    {\if@restonecol\twocolumn \else \newpage \fi
    }
\else
  \renewenvironment{titlepage}
    {%
      \if@twocolumn
        \@restonecoltrue\onecolumn
      \else
        \@restonecolfalse\newpage
      \fi
      \thispagestyle{empty}%
      % \setcounter{page}\@ne
    }%
    {\if@restonecol\twocolumn \else \newpage \fi
     \if@twoside\else
        % \setcounter{page}\@ne
     \fi
    }
\fi
\makeatother

% First pages are numbered A, B, C, ...
% Also, this avoids wrong back references with the biblatex package.
\pagenumbering{Alph}

\begingroup
% Use Helvetica font in the first pages (according to the UA rules).
% https://tex.stackexchange.com/questions/427245/how-to-use-helvetica-font-in-online-editor
% https://www.overleaf.com/learn/latex/Font_typefaces
% In fact the TeX Gyre Heros is used because it can be used as a
% substitute for Adobe Helvetica.

\ifPDFTeX
\renewcommand{\sfdefault}{qhv}
% \renewcommand{\sfdefault}{phv}
\fi

\ifLuaTeX
% \renewfontfamily\sffamily{Arial}
% \renewfontfamily\sffamily{Arimo}
% \renewfontfamily\sffamily{Helvetica}
\renewfontfamily\sffamily{TeX Gyre Heros}
\fi

% Cover page.
\TitlePage
  \HEADER{\BAR}{\thesisyear}
  \vspace*{14mm}
  \TITLE{\authorname}{\portuguesetitle}
  \vspace*{7mm}
  \TITLE{}{\englishtitle}
\EndTitlePage

% Empty page.
% \titlepage\ \endtitlepage

% Initial thesis pages.
\TitlePage
  \HEADER{}{\thesisyear}
  \vspace*{14mm}
  \TITLE{\authorname}{\portuguesetitle}
  \vspace*{7mm}
  \TITLE{}{\englishtitle}
  \vspace*{15mm}
  \TEXT{}{Dissertação/Tese apresentada à Universidade de Aveiro para cumprimento dos requisitos necessários à obtenção do grau de Mestre/Doutor em X, realizada sob a orientação científica de Y, Professor do Departamento Z da Universidade de Aveiro.}
  \vspace*{\fill}
  \TEXT{}{Apoio financeiro da FCT e do FSE no âmbito do III Quadro Comunitário de Apoio. (se aplicável)}
\EndTitlePage

% Empty page.
% \titlepage\ \endtitlepage

% Dedication text (optional).
\TitlePage

\vspace*{81.5mm}
\TEXT{}{À minha família e amigos, pelo vosso amor. (dedicação, opcional)}
\vspace*{12mm}
\TEXT{}{To my family and friends, for your love. (dedication, optional)}

%% Or, do something different as you prefer such as:
% \vspace*{81.5mm}
% \TEXT{\textbf{dedicação}}{À minha família e amigos, pelo vosso amor. (opcional)}
% \vspace*{12mm}
% \TEXT{\textbf{dedication}}{To my family and friends, for your love. (optional)}

%% Or:
% \vspace*{55mm}
% \begin{center}
% \fontsize{14}{16.8}\selectfont
% \textit{À minha família e amigos, pelo vosso amor.}\\[12pt]
% \textit{To my family and friends, for your love.}
% \end{center}

\EndTitlePage

% Empty page.
% \titlepage\ \endtitlepage

\TitlePage
  \vspace*{81.5mm}
  \TEXT{\textbf{o júri~/~the jury\newline}}
       {}
  \TEXT{\small presidente~/~president}
       {\textbf{ABC}\newline Professor Catedrático da Universidade de Aveiro}
  \vspace*{5mm}
  \TEXT{\small vogais~/~examiners committee}
       {\textbf{DEF}\newline Professor Catedrático da Universidade de Aveiro (orientador)}
  \vspace*{5mm}
  \TEXT{}
       {\textbf{GHI}\newline Professor Associado da Universidade J (co-orientador)}
  \vspace*{5mm}
  \TEXT{}
       {\textbf{KLM}\newline Professor Catedrático da Universidade N}
\EndTitlePage

% Empty page.
% \titlepage\ \endtitlepage

\TitlePage
  \vspace*{81.5mm}
  \TEXT{\textbf{agradecimentos}}
       {... (opcional)}
  \vspace*{12mm}
  \TEXT{\textbf{acknowledgments}}
       {... (optional)}
\EndTitlePage

% Empty page.
% \titlepage\ \endtitlepage

\TitlePage
  \vspace*{81.5mm}
  \TEXT{\textbf{palavras-chave}}{Palavra $\cdot$ Chave.}
  \vspace*{12mm}
  \TEXT{\textbf{resumo}}{Este é o primeiro parágrafo do resumo.\newline Segundo parágrafo.\newline Terceiro parágrafo.}
\EndTitlePage

% Empty page.
% \titlepage\ \endtitlepage

\TitlePage
  \vspace*{81.5mm}
  \TEXT{\textbf{keywords}}{Key $\cdot$ Word.}
  \vspace*{12mm}
  \TEXT{\textbf{abstract}}{This is the first paragraph of the abstract.\newline Second paragraph.\newline Third paragraph.}
\EndTitlePage

% Empty page.
% \titlepage\ \endtitlepage

% How to add an empty line between paragraphs.
% https://tex.stackexchange.com/questions/135134/how-to-add-an-empty-line-between-paragraphs
\TitlePage
  \vspace*{81.5mm}
  \TEXT{\textbf{reconhecimento do uso de ferramentas de IA}}{%
    \textbf{Reconhecimento do uso de tecnologias e ferramentas de Inteligência}\\
    \textbf{Artificial generativa, programas, e outras ferramentas de apoio.}\\[0.8\baselineskip]
    Reconheço o uso de [\textit{inserir sistema(s) de IA e respetiva(s) hiperligação(ões)}] para [\textit{indicar utilização específica de IA ou outras tarefas}]. Reconheço a utilização de [\textit{indicar programa, código, ou plataforma}] para [\textit{indicar utilização específica do programa, código, ou plataforma, ou outras tarefas}].\\[\baselineskip]
    Exemplo 1:\\
    Reconheço a utilização do ChatGPT 3.5 (OpenAI, \href{https://chatgpt.com}{\texttt{chatgpt.com}}) para resumir as notas iniciais e para rever o rascunho final.\\[\baselineskip]
    Exemplo 2:\\
    Não foram utilizados no presente documento quaisquer conteúdos gerados por tecnologias de IA.
  }
  \vspace*{12mm}
  \TEXT{\textbf{acknowledgment of use of AI tools}}{%
    \textbf{Recognition of the use of generative Artificial Intelligence technologies}\\
    \textbf{and tools, software, and other support tools.}\\[0.8\baselineskip]
    I acknowledge the use of [\textit{insert AI system(s) and link(s)}] to [\textit{specific use of generative artificial intelligence or other tasks}]. I acknowledge the use of [\textit{software, code, or platform}] to [\textit{specific use of sofware, code, or platform, or other tasks}].\\[\baselineskip]
    Example 1:\\
    I acknowledge the use of ChatGPT 3.5 (OpenAI, \href{https://chatgpt.com}{\texttt{chatgpt.com}}) to summarize the initial notes and to proofread the final draft.\\[\baselineskip]
    Example 2:\\
    No content generated by AI technologies has been used in this document.
  }
\EndTitlePage

% Empty page.
% \titlepage\ \endtitlepage

% End of Helvetica font.
\endgroup


% Table of contents (mandatory for compiling without errors).
% To specify 1x or 1.5x vertical spacing between lines.
% \singlespacing
\onehalfspacing

% Tables of contents, of figures, of tables.

% To count the following pages with roman numbering.
\pagenumbering{roman}

\tableofcontents

\listoffigures

\listoftables

\begingroup
% To locally reduce vertical space between entries.
\setlist{itemsep=0pt,topsep=0pt,parsep=0pt,partopsep=0pt}
\printnoidxglossary
\endgroup

% To specify 1x or 1.5x vertical spacing between lines.
% \singlespacing
\onehalfspacing

% The chapters.

% To count the following pages with Arabic numerals.
\pagenumbering{arabic}


% Notation, nomenclature, or list of symbols (can be commented).
% This file allows the author to manually create a "Notation",
% "Nomenclature", or "List of symbols" section, by having full control
% and design freedom. Also, it has no special package dependencies.
%
% This "notation.tex" file was inspired from other works:
%
% [1] Tiago Almeida (2025).
%     Biomedical question answering: from retrieval to answer generation.
%     http://hdl.handle.net/10773/44153
%
% [2] Ian Goodfellow, Yoshua Bengio, and Aaron Courville (2016).
%     Deep Learning.
%     https://www.deeplearningbook.org
%     https://github.com/goodfeli/dlbook_notation/
%
% [3] Yarin Gal (2016).
%     Uncertainty in deep learning.
%     https://idiscover.lib.cam.ac.uk/permalink/f/t9gok8/44CAM_ALMA21582084170003606
%
% [4] Simon Baker (2017).
%     Semantic text classification for cancer text mining.
%     https://doi.org/10.17863/CAM.23105
%
% Different ways (more or less complex, and more or less automatic) of
% achieving a similar result do exist, but are not implemented in this
% template. For example, see:
%   https://tex.stackexchange.com/questions/348640/how-to-effectively-use-list-of-symbols-for-a-thesis
%   https://www.overleaf.com/learn/latex/Nomenclatures

\begingroup
% Set the glue (vertical space) before a "longtable" to be 0pt.
% https://tex.stackexchange.com/questions/424176/longtable-vertical-space
\setlength{\LTpre}{0pt}

% Make the "longtable" environment to be left-aligned by default.
% https://stackoverflow.com/questions/3345077/latex-left-align-a-table-ie-not-centred-from-the-preamble
\setlength{\LTleft}{0pt}
\setlength{\LTright}{\fill}

% Set the "tocdepth" counter to 0 so only the chapter-level is shown
% at the table of contents.
% https://tex.stackexchange.com/questions/4102/setting-setcountertocdepth-for-an-individual-chapter
\addtocontents{toc}{\protect\setcounter{tocdepth}{0}}
% Set the "secnumdepth" counter to 0 so no section level is numbered.
\setcounter{secnumdepth}{0}

\phantomsection
\addcontentsline{toc}{chapter}{\notationname}
\chapter*{\notationname}
\markboth{\notationname}{}
\label{notation}

This part of the template can be edited to manually create a \textbf{Notation}, \textbf{Nomenclature}, or \textbf{List of symbols} section, as the author may prefer.
I leave below a draft of what it may look like.


\section{Physical constants}
\label{notation:s:physical-constants}
% Adapted from:
% https://www.overleaf.com/learn/latex/Nomenclatures

\begin{longtable}{llr}
$G$ & Gravitational constant     & \SI[group-digits=false]{6.67430e-11}{\meter\cubed\per\kilogram\per\second\squared}\\
$c$ & Speed of light in a vacuum & \SI{299792458}{\meter\per\second}\\
$h$ & Planck constant            & \SI[group-digits=false]{6.62607015e-34}{\joule\per\hertz}\\
\end{longtable}


\section{Mathematical symbols}
\label{notation:s:mathematical-symbols}
% Adapted from:
% https://en.wikipedia.org/wiki/Glossary_of_mathematical_symbols


\subsection{Arithmetic operators}
\label{notation:ss:arithmetic-operators}

\begin{longtable}{ll}
$+$      & plus sign\\
$-$      & minus sign\\
$\times$ & multiplication sign\\
\end{longtable}


\subsection{Equality}
\label{notation:ss:equality}

\begin{longtable}{ll}
$=$       & equality\\
$\neq$    & inequality\\
$\approx$ & approximate equality\\
\end{longtable}


\subsection{Comparison}
\label{notation:ss:comparison}

\begin{longtable}{ll}
$<$    & less-than sign\\
$>$    & greater-than sign\\
$\leq$ & less than or equal to\\
$\geq$ & greater than or equal to\\
\end{longtable}


\section{Numbers}
\label{notation:s:numbers}

\begin{longtable}{ll}
$a$         & A scalar\\
$\vec{a}$   & A vector\\
$A$         & A matrix\\
\end{longtable}


\section{Operations}
\label{notation:s:operations}
% https://tex.stackexchange.com/questions/107186/how-to-write-norm-which-adjusts-its-size
% https://tex.stackexchange.com/questions/30619/what-is-the-best-symbol-for-vector-matrix-transpose

\begin{longtable}{ll}
$|\vec{a}|$                      & $L_1$ norm of vector $\vec{a}$\\
$\left\lVert\vec{a}\right\rVert$ & $L_2$ norm of vector $\vec{a}$\\
$A^\top$                         & Transpose of matrix $A$\\
\end{longtable}


\section{Other symbols}
\label{notation:s:other-symbols}

Add here some other symbols as needed.


\section{Miscellaneous}
\label{notation:s:miscellaneous}

Other typography rules, conventions, or definitions.


% The \cleardoublepage command is necessary to make sure that the
% page numbering remains correct.
\cleardoublepage
% Reset the "tocdepth" and "secnumdepth" counters to the original
% values.
\addtocontents{toc}{\protect\setcounter{tocdepth}{\mytocdepth}}
\setcounter{secnumdepth}{\mysecnumdepth}

\endgroup


% To count the following pages with Arabic numerals.
\pagenumbering{arabic}

% Chapters (can be commented).
\chapter{The package}
\label{c1}

This chapter presents briefly how to use the \verb+uathesis+ package.


\section{UA thesis \LaTeX\ style file}
\label{c1:s:uathesis-style-file}

The \verb+uathesis+ \LaTeX\ style file was originally created by professor Tomás Oliveira e Silva~\parencite*{oliveiraesilva2012a}, and it is currently available at his home page\footnote{\url{http://sweet.ua.pt/tos/TeX.html}}.
In this template it is used a new version modified by Rui Antunes\footnote{\url{https://github.com/ruiantunes/ua-thesis-template}}.


\subsection{Options}
\label{c1:ss:options}

The following options are supported:

\begin{itemize}

\item
\texttt{oldLogo}: to use the old logo of University of Aveiro. If not specified, the new logo is used by default.

\item
\texttt{MAP}: for MAP joint doctoral programmes. The logos from the three universities (Aveiro, Minho, Porto) are used.

\item
\texttt{draft}: it prints ``DOCUMENTO PROVISÓRIO'' in the first two front pages.

\item
\texttt{draftPT}: same as \texttt{draft}.

\item
\texttt{draftEN}: same as \texttt{draft}, but instead it prints ``DRAFT DOCUMENT''.

\item
As of May 29, 2021, the department name shall not appear in the first pages (top header).
A new option, \texttt{NODEPT} (no department), was created to suppress the department name (now this is the default behavior).\\
However, formerly the department name would appear in the cover and first page, therefore the old options were kept for the sake of preservation.
Any department name can be shown by using one of the following options: \texttt{DAO}, \texttt{DBIO}, \texttt{DCM}, \texttt{DCSPT}, \texttt{DECA}, \texttt{DECIVIL}, \texttt{DEGEIT}, \texttt{DEM}, \texttt{DEMAC}, \texttt{DEP}, \texttt{DETI}, \texttt{DFIS}, \texttt{DGEO}, \texttt{DLC}, \texttt{DMAT}, \texttt{DQ}.

\item
The color of the top bar, in the cover page, is defined by specifying one of the following scientific areas: \texttt{accounting}, \texttt{arts}, \texttt{arts+humanities}, \texttt{economy}, \texttt{education}, \texttt{engineering}, \texttt{health}, \texttt{humanities}, \texttt{misc}, \texttt{sciences}.

\item
As of October 28, 2024, the horizontal rule shall not appear in the first pages (top header).
A new option, \texttt{horizontalRule}, was created to keep the old behavior for the sake of preservation. By not specifying this option the default behavior is applied, that is, no horizontal rule is drawn.

\end{itemize}

\chapter{Tips and examples}
\label{c2}

This chapter presents some basic tips and a few examples on how to use \LaTeX.


\section{\TeX\ engines}
\label{c2:s:tex-engines}

There are several \TeX\ engines.
In short, these are used to compile the (La)TeX source code to generate the output file (for example, a \as{pdf}).
To know more about these, I suggest you to check these articles:

\begin{itemize}
\item
The TeX family tree: LaTeX, pdfTeX, XeTeX, LuaTeX and ConTeXt.\\
\url{https://www.overleaf.com/learn/latex/Articles/The_TeX_family_tree:_LaTeX,_pdfTeX,_XeTeX,_LuaTeX_and_ConTeXt}

\item
Choosing a LaTeX Compiler.\\
\url{https://www.overleaf.com/learn/latex/Choosing_a_LaTeX_Compiler}

\item
Are there benefits to use XeTeX or LuaTeX if one is to write documents mainly in English?\\
\url{https://tex.stackexchange.com/questions/548467/are-there-benefits-to-use-xetex-or-luatex-if-one-is-to-write-documents-mainly-in}

\item
Why choose LuaLaTeX over XeLaTeX.\\
\url{https://tex.stackexchange.com/questions/126206/why-choose-lualatex-over-xelatex}

\item
Differences between LuaTeX, ConTeXt and XeTeX.\\
\url{https://tex.stackexchange.com/questions/36/differences-between-luatex-context-and-xetex}

\end{itemize}

In this template, support for both pdf\TeX\ and Lua\TeX\ engines has been guaranteed, but I encourage you to use the latter because it is more powerful for typefaces: it supports TrueType and OpenType standards.


\subsection{Compiler automatic detection}

Automatic detection of the \LaTeX\ compiler in use:

\ifPDFTeX
{\color{red} pdf\TeX\ is being used.}

Consider changing to Lua\TeX, which is the recommended compiler for this template.
\fi

\ifLuaTeX
{\color{Green4} Lua\TeX\ is being used.}

You are using the recommended compiler.
\fi


\section{Basic tips}
\label{c2:s:basic-tips}

\begin{itemize}
\item
Use the \verb+%+ (percentage) symbol to comment (ignore) lines in the source code.
\item
The \verb+\\+ and \verb+\newline+ commands are similar, but are not the same.\\
\url{https://tex.stackexchange.com/questions/27028/what-is-the-difference-between-newline-and}.
\item
The \verb+\cleardoublepage+ command forces the next content to start in an odd page.
\item
The tilde character (\verb+~+) inserts a non-breaking space.
Use it before citing a reference to avoid breaking the line: \verb+an example~\cite{label}+.
\item
The current font size is \myfontsize.
\item
Use the \verb+longtable+ environment for tables spanning multiple pages.
\item
The grave accent \`{} and the apostrophe \textquotesingle\ are the correct symbols to make quotations: ``this is an example''.
\end{itemize}


\section{How to compile the document faster}
\label{c2:s:how-to-compile-the-document-faster}

Some techniques can be employed to compile the document faster (note that these do not apply for generating the final document).
In the \verb+main.tex+ file:

\begin{itemize}
\item Use the \verb+draft+ option in the \verb+graphicx+ package (images will be replaced by placeholders of the same size);
\item Comment the \verb+first-pages.tex+ and \verb+notation.tex+ files;
\item Compile only the chapter that you're currently working with (the other chapters can be commented);
\item You may also comment the references, appendices, and annexes.
\end{itemize}


\section{The chapter style}
\label{c2:s:the-chapter-style}

Using the \verb+titlesec+ package you can build your own chapter style.
Instead, for ease of use, you may use one of the styles that are already provided in this template:

\begin{itemize}

\item
Default style:\\
\verb+% Adapted from:
% https://tug.ctan.org/macros/latex/contrib/titlesec/titlesec.pdf#subsection.8.2

\titleformat{\chapter}[display]
  {\normalfont\huge\bfseries}{\chaptertitlename\ \thechapter}{20pt}{\Huge}

\titlespacing*{\chapter}{0pt}{50pt}{40pt}
+

\item
Veelo chapter style:\\
\verb+% Veelo chapter style. Adapted from:
% https://tex.stackexchange.com/questions/487917/setting-chapter-style-with-bar
% https://tex.stackexchange.com/questions/177144/chapter-style-report-class
% Note that you can change \rule{30pt} to another width (such as 28pt
% or 32pt if you prefer a thinner or thicker bar).
\newcommand*\chapnamefont{\normalfont\LARGE\MakeUppercase}
\newcommand*\chapnumfont{\normalfont\Huge}
\newcommand*\chaptitlefont{\normalfont\Huge\bfseries}

\newlength\beforechapskip
\newlength\midchapskip
\setlength\midchapskip{\paperwidth}
\addtolength\midchapskip{-\textwidth}
\addtolength\midchapskip{-\oddsidemargin}
\addtolength\midchapskip{-1in}
% Also adjust this value accordingly to your preference.
\setlength\beforechapskip{17mm}

\titleformat{\chapter}[display]
  {\normalfont\filleft}
  {{\chapnamefont\chaptertitlename}%
    \rlap{\hspace{.8em}%
      \makebox[\dimexpr\oddsidemargin+\hoffset+1in][s]{{\resizebox{!}{\beforechapskip}{\chapnumfont\thechapter}}\hfill%
      \rule{30pt}{\beforechapskip}}%
    }%
  }%
  {25pt}
  {\chaptitlefont}
\titlespacing*{\chapter}
  {0pt}{40pt}{40pt}
+

\item
Centered title within two rules:\\
\verb+% Adapted from:
% https://pt.overleaf.com/learn/latex/Sections_and_chapters
\titleformat
  {\chapter} % command
  [display] % shape
  {\bfseries\LARGE} % format
  {\chaptertitlename\ \thechapter} % label
  {0.5ex} % sep
  {
    \rule{\textwidth}{1pt}
    \vspace{1ex}
    \centering
    \huge
  } % before-code
  [
    \vspace{-2.0ex}%
    \rule{\textwidth}{1pt}
  ] % after-code
+

\item
Left-aligned title within two rules (no chapter number):\\
\verb+\titleformat{name=\chapter}[display]
  {\bfseries\Large}
  {}
  {1ex}
  {\titlerule\vspace{2ex}\filright\huge}
  [\vspace{1ex}\titlerule]
+

\item
Title with a bottom rule:\\
\verb+\titleformat{\chapter}
  {\normalfont\Huge\bfseries}
  {\thechapter.}
  {20pt}
  {\Huge}
  [\titlerule]
+

\item
Centered title within two rules (but unnumbered chapters have no rules):\\
\verb+% Adapted from:
% https://tex.stackexchange.com/questions/77788/chapter-titles-using-titlesec
\titleformat{\chapter}[display]
  {\normalfont\bfseries\filcenter}{\LARGE\thechapter}{1ex}
  {\titlerule[2pt]\vspace{2ex}\LARGE}[\vspace{1ex}{\titlerule[2pt]}]

\titleformat{name=\chapter,numberless}[display]
  {\normalfont\LARGE\bfseries\filcenter}{}{1ex}
  {\vspace{2ex}}[\vspace{1ex}]
+

\end{itemize}

You can make this selection in the \verb+tex/config/packages.tex+ file.


\section{Font types}
\label{c2:s:font-types}

You may change the font types in use in the \verb+tex/config/packages.tex+ file.
If you're trying to use different font types make sure you have them installed in your system.


\section{Font styles}
\label{c2:s:font-styles}

\begin{itemize}
\item
\verb+\textnormal{}+ --- \textnormal{document font family}.
\item
\verb+\textrm{}+ --- \textrm{roman font family}.
\item
\verb+\textsf{}+ --- \textsf{sans serif font family}.
\item
\verb+\texttt{}+ --- \texttt{teletypefont family}.
\item
\verb+\textit{}+ --- \textit{italic shape}.
\item
\verb+\textsl{}+ --- \textsl{slanted shape}.
\item
\verb+\textsc{}+ --- \textsc{small capitals}.
\item
\verb+\textbf{}+ --- \textbf{bold}.
\end{itemize}


\section{Colors}
\label{c2:s:colors}

An example of {\color{red} red colored text} from the \texttt{color} package.
And {\color{Blue4} Blue4 colored text} from the \texttt{xcolor} package.

\section{Footnotes}
\label{c2:s:footnotes}

This is a labeled footnote\cref{foot:example}.
A footnote can be referenced multiple times\footnote{\label{foot:example}This is a footnote example.}.
Again, the same footnote is referenced\cref{foot:example}.


\subsection{A table example}
\label{c2:ss:a-table-example}

A table example is shown in \Cref{tab:a-table-example}.

\FloatBarrier
\begin{table}[!htbp]
\caption{A table example.}
\label{tab:a-table-example}
\centering
\begin{tabular}{m{33.3mm}D{33.3mm}F{33.3mm}G{33.3mm}N}
\toprule
justified  & left-aligned & centered   & right-aligned & \\
\midrule
\dummytext & \dummytext   & \dummytext & \dummytext    & \\
\midrule
This is an example & 2 & 3 & 4 & \\
\midrule
Single cell & \multicolumn{3}{|c|}{A multi-column cell} & \\
\midrule
\multirow{2}[1]{*}{A multi-row cell} & A simple & ... & ... & \\
& example & ... & ... & \\
\bottomrule
\end{tabular}
\end{table}
\FloatBarrier



\section{Abbreviations}
\label{c2:s:abbreviations}

\verb+\gls{label}+ and \verb+\glslink{label}{text}+ are two possible commands for making use of abbreviations.
For example, the commands \verb+\gls{afk}+ (first call), \verb+\gls{afk}+ (second call), and \verb+\glslink{afk}{insert specific text}+ produce respectively ``\gls{afk}'', ``\gls{afk}'' and ``\glslink{afk}{insert specific text}''.

A list of some commands follow.

\begin{itemize}
\item
\verb+\gls{afk}+ produces ``\gls{afk}''.
\item
\verb+\glslink{afk}{text}+ produces ``\glslink{afk}{text}''.
\item
\verb+\glsxtrshort{afk}+ and \verb+\as{afk}+ produce ``\glsxtrshort{afk}'' and ``\as{afk}'', respectively.
\item
\verb+\glsxtrlong{afk}+ and \verb+\al{afk}+ produce ``\glsxtrlong{afk}'' and ``\al{afk}'', respectively.
\end{itemize}

Note that the commands \verb+\as{}+ and \verb+\al{}+ are shorter variants.

Other abbreviations include: good work (\as{gw}); have fun (\as{hf}); good work and have fun (\as{gwhf}).
Note that the latter contains nested abbreviations.


\section{Equations}
\label{c2:s:equations}

\Cref{eq:example1} is a numbered equation.

\begin{equation}
\label{eq:example1}
x = 1 + y
\end{equation}

The following equation is not numbered, and thus cannot be referenced.

\begin{equation*}
y = \sum_{i=1}^{N}{x_i}
\end{equation*}

The \verb+\myscore+ pre-defined math function is used in \Cref{eq:example2}.

\begin{equation}
\label{eq:example2}
\myscore(d) = \frac{1}{d^2}
\end{equation}


\section{Figures}
\label{c2:s:figures}

An example of a figure is shown in \Cref{fig:aveiro}. The \verb+\fbox+ command \fbox{draws a box} around its content.

\input{tex/contents/fig/aveiro.tex}


\section{Rotating pages}
\label{c2:s:rotating-pages}

Pages can be displayed horizontally (landscape orientation) using the \verb+landscape+ environment  from the \verb+pdflscape+ package as:

\begin{verbatim}
\begin{landscape}
Add some text or figures.
\end{landscape}
\end{verbatim}


\section{A long table}
\label{c2:s:a-long-table}

\Cref{tab:a-long-table} presents a long table using the \verb+longtable+ package.
This table can span multiple pages.
The \verb+\afterpage+ command forces the table to start at the top of a page.

\afterpage{\input{tex/contents/tab/a-long-table.tex}}\cleardoublepage

\chapter{How to reference}
\label{c3}

This chapter presents how to reference (1) documents such as books, journal articles, conference articles, among other works and (2) elements of this document, such as, chapters, sections, subsections, equations, figures, and tables.


\section{Citing other works}
\label{c3:s:citing-other-works}

In this template, it is used the \verb+biblatex+ package for citing other works.
It is important to note that Bib\TeX\ and Bib\LaTeX\ have many similarities but are two different formats.
For example, if you want to indicate a conference's place you have to use the \verb+address+ field in Bib\TeX, but the \verb+venue+ field in Bib\LaTeX.
The \verb+address+ field in Bib\LaTeX\ is used to indicate the publisher's local. If you would like to indicate the publisher in journal articles I suggest you to use the \verb+note+ field.
By default, the references are expected to be in the \verb+refs.bib+ file.
Organize your citations with a reference manager such as, for example, Zotero\footnote{\url{https://www.zotero.org/}} with the Better Bib\TeX\ extension\footnote{\url{https://retorque.re/zotero-better-bibtex/}}.
Zotero is what I personally use and recommend.
For my citation key labels I use the format \texttt{[author][year][a]}: lowercase string with the last name(s) of the first author followed by the year of the publication and a suffix letter for distinction of repeated works (check the \texttt{refs.bib} file for some examples).


\subsection{The citation style}
\label{c3:ss:the-citation-style}

There are several citation styles.
You can configure your own, or use one of the three configurations that are already provided in this template:

\begin{itemize}
\item
Author-year style:\\\verb+\input{tex/config/biblatex/authoryear.tex}+.
\item
Author-year compact style:\\\verb+\input{tex/config/biblatex/authoryear-comp.tex}+
\item
IEEE numeric style:\\\verb+% BibLaTeX numeric style.
\usepackage[%
backend=biber,%
% Print back references in the bibliography.
% backref=false,%
backref=true,%
bibstyle=ieee,%
citestyle=numeric-comp,%
dashed=false,%
defernumbers=true,%
hyperref=true,%
maxbibnames=999,%
minbibnames=1,%
maxcitenames=2,%
mincitenames=1,%
sorting=none,%
sortcites=false,%
]{biblatex}
+.
\item
Numeric style:\\\verb+\input{tex/config/biblatex/numeric.tex}+.
\end{itemize}

You can choose one of these in the \verb+tex/config/packages.tex+ file.
By default, the \verb+authoryear-comp+ style is selected.


\subsection{The citation commands}
\label{c3:ss:the-citation-commands}

A list of citation commands, and examples, follows:

\begin{itemize}

\item
\verb+\cite{oliveiraesilva2012a}+ produces:\\
\cite{oliveiraesilva2012a}

\item
\verb+\cite{oliveiraesilva2012a,oliveiraesilva1994a}+ produces:\\
\cite{oliveiraesilva2012a,oliveiraesilva1994a}

\item
\verb+\textcite{oliveiraesilva2012a}+ produces:\\
\textcite{oliveiraesilva2012a}

\item
\verb+\textcite{oliveiraesilva2012a,oliveiraesilva1994a}+ produces:\\
\textcite{oliveiraesilva2012a,oliveiraesilva1994a}

\item
\verb+\parencite{oliveiraesilva2012a}+ produces:\\
\parencite{oliveiraesilva2012a}

\item
\verb+\parencite{oliveiraesilva2012a,oliveiraesilva1994a}+ produces:\\
\parencite{oliveiraesilva2012a,oliveiraesilva1994a}

\item
\verb+\parencite*{oliveiraesilva2012a}+ produces:\\
\parencite*{oliveiraesilva2012a}

\item
\verb+\parencite*{oliveiraesilva2012a,oliveiraesilva1994a}+ produces:\\
\parencite*{oliveiraesilva2012a,oliveiraesilva1994a}

\item
\verb+\fullcite{oliveiraesilva2012a}+ produces:\\
\fullcite{oliveiraesilva2012a}

\end{itemize}

The \verb+\cite{label}+ command is preferred with the numeric style.
In the \texttt{authoryear} and \texttt{authoryear-comp} styles, the \verb+\textcite{label}+ and \verb+parencite{label}+ commands should be used.

Multiple citations are allowed by using multiple labels within the same command.
Example: \verb+\cite{oliveiraesilva2012a,oliveiraesilva1994a}+.

More examples follow.
A \as{phd} thesis~\parencite{oliveiraesilva1994a}.
A \as{msc} dissertation~\parencite{antunes2015a}.
Conference or workshop articles~\parencite{antunes2019a,antunes2020a}.
A journal article~\parencite{antunes2019b}.


\section{Referencing elements of this document}
\label{c3:s:referencing-elements-of-this-document}

\Cref{c1,c3} are two chapters.

\Cref{c1,c2,c3} are three consecutive chapters.

\Cref{c1,c3,c4} are three non-consecutive chapters.

\Cref{c3:s:citing-other-works,c3:s:a-section} are two sections.

\Cref{eq:example1,eq:example2} are two numbered equations.

\Cref{fig:aveiro} is a figure.

\Cref{tab:a-table-example} is a table.

\Cref{c3:s:a-section,c3:ss:a-subsection} have different levels of depth.

\Cref{ap1,ap2} are two appendices.

\Cref{an1} is an annex.

\Cref{an1,an2} are two annexes.


\section{A section}
\label{c3:s:a-section}

This section presents the different levels of depth.


\subsection{A subsection}
\label{c3:ss:a-subsection}

Text in a subsection.


\subsubsection{A subsubsection}

Text in a subsubsection.


\myparagraph{A paragraph}

Text in a paragraph.


\mysubparagraph{A subparagraph}

Text in a subparagraph.

\input{tex/contents/c4.tex}

% References (can be commented).
% To properly end the last page of the last chapter, so that the page
% number of the bibliography is correct (in the table of contents).
\cleardoublepage

% Bibliography.

% To cite all the references (debugging purposes).
% \nocite{*}

% The bibliography using BibLaTeX.
\phantomsection
\addcontentsline{toc}{chapter}{\bibname}
\printbibliography


% Appendices (can be commented).
% Difference between annex and appendix.
% https://ell.stackexchange.com/questions/321470/difference-between-annex-and-appendix

\renewcommand*{\appendixname}{\myappendixname}
\renewcommand*{\appendixtocname}{\myappendixtocname}
\renewcommand*{\appendixpagename}{\myappendixpagename}

\begin{appendices}
\crefalias{chapter}{appendix}

% To enumerate the Appendices with Arabic numbers.
% \renewcommand{\thechapter}{\arabic{chapter}}
% \renewcommand{\thesection}{\arabic{chapter}.\arabic{section}}
% \renewcommand{\thesubsection}{\arabic{chapter}.\arabic{section}.\arabic{subsection}}

% To enumerate the Appendices with Roman numbers.
% \renewcommand{\thechapter}{\Roman{chapter}}
% \renewcommand{\thesection}{\Roman{chapter}.\arabic{section}}
% \renewcommand{\thesubsection}{\Roman{chapter}.\arabic{section}.\arabic{subsection}}

\chapter{Appendix example}
\label{ap1}

This is the first appendix.
Note that appendices and annexes may serve different purposes.


\section{A section example}
\label{ap1:s:a-section-example}

Similarly to a chapter we can add sections, subsections, and so on...

\chapter{A second example of an appendix}
\label{ap2}

This is the second appendix.


\end{appendices}


% Annexes (can be commented).
% Difference between annex and appendix.
% https://ell.stackexchange.com/questions/321470/difference-between-annex-and-appendix

\renewcommand*{\appendixname}{\myannexname}
\renewcommand*{\appendixtocname}{\myannextocname}
\renewcommand*{\appendixpagename}{\myannexpagename}

\begin{appendices}
\crefalias{chapter}{annex}

% To enumerate the Annexes with numbers.
\renewcommand{\thechapter}{\arabic{chapter}}
\renewcommand{\thesection}{\arabic{chapter}.\arabic{section}}
\renewcommand{\thesubsection}{\arabic{chapter}.\arabic{section}.\arabic{subsection}}

\chapter{Annex example}
\label{an1}

This is the first annex.
Note that appendices and annexes may serve different purposes.


\section{A section example}
\label{an1:s:a-section-example}

Similarly to a chapter we can add sections, subsections, and so on...


\subsection{A subsection example}
\label{an1:ss:a-subsection-example}

A subsection in an annex.


\subsubsection{A subsubsection example}
\label{an1:sss:a-subsubsection-example}

A subsubsection in an annex.

\chapter{A second example of an annex}
\label{an2}

This is the second annex.


\end{appendices}


% To make sure the document ends with an even number page.
\cleardoublepage

\end{document}
